
Many programming models have been developed in order to make parallel programs both easier to write and reason about. 
Declarative programming paradigms move away from the pitfalls of imperative programming and transfer the problem of parallelizing
the program to the compiler and runtime system. Famous examples of such paradigms are \emph{logic programming} and \emph{functional programming}.
In logic languages such as Prolog, researchers took advantage of the non-determinism of proof-search to evaluate subgoals
in parallel with models such as \emph{or-parallelism}~\cite{ali-86} and \emph{and-parallelism}~\cite{Shen-92}.
In functional languages, the stateless nature of computation allows multiple expressions to evaluate safely in parallel.
This has been explored in several languages such as NESL~\cite{Blelloch:1996:PPA:227234.227246} or Id~\cite{Nikhil93anoverview}.

Recently, there has been an increasing interest in declarative and data-centric languages.
MapReduce~\cite{Dean:2008:MSD:1327452.1327492}, for instance, is a popular data-centric programming
model that is optimized for large clusters. The scheduling and data sharing model is very simple:
in the \emph{map phase}, data is transformed at each node and the result reduced to a final
result in the \emph{reduce phase}.

A declarative approach that is regaining popularity is Datalog~\cite{Ullman:1990:PDK:533142}, a
bottom-up logic programming language.
Traditionally used in deductive databases, Datalog is being increasingly used in different fields
such as distributed networking~\cite{Loo-condie-garofalakis-p2}, sensor
nets~\cite{Chu:2007:DID:1322263.1322281} and cloud computing~\cite{alvaro:boom}.

Like \lang, many programming systems also model the program as a graph where computation will be performed.
The Dryad system~\cite{Isard:2007:DDD:1272996.1273005} combines computational vertices
with communication channels (edges) to form a data-flow graph. The program is scheduled to
run on multiple computers or cores and data is partitioned during runtime. Routines that run on computational vertices
are sequential, with no locking required. Dryad is design to scale in number of available resources.

The Pregel system~\cite{Malewicz:2010:PSL:1807167.1807184} is also graph based, although programs have a more strict
structure. They must be represented as a sequence of iterations where each iteration is composed of computation and message passing.
Pregel is specially suited for solving very big graphs
and is also intended to scale to large architectures.

GraphLab~\cite{GraphLab2010} is a C++ framework for developing parallel machine learning algorithms. While
Pregel uses message passing, GraphLab allows nodes to have read/write access to different scopes through different concurrent access models in order to balance performance and data consistency. While some programs only need to access the local node's data, others may need to update edge information. Each consistency model will provide different guarantees that are better adapted to some algorithms. GraphLab also provides different schedulers that dictate the order in which node's are computed. Later in this paper, we will show how certain GraphLab's schedulers can be easily implemented in \lang through the use of coordination facts.
