Parallel programs are known to be difficult to write and reason about. Proving that a parallel
program is correct is made difficult by the constraints of imperative programming and the low
level constructs that are available with such paradigms. Declarative
programming is a step forward in the right direction but it tends to remove
too much control from the programmer.
We have designed a new declarative logic programming language that attempts to solve
this problem. The foundation of our language is linear logic, a powerful logical system where logical
facts can be removed, which allows writing programs that express state
while still remaining declarative. When running a linear logic program, appropriate
scheduling strategies may result in faster execution times, therefore we introduce
logical rules that can guide the coordination of programs. The rules are written as regular rules,
but make use of sensing and action facts to sense execution and coordinate, respectively.
We have written graph algorithms, machine learning algorithms and several other programs and have
seen good results on multicores, although our runtime system can be easily
extended to other distributed architectures.