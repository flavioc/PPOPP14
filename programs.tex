We now describe how some programs are implemented in \lang.
\begin{comment}
We will present a graph algorithm
(single source shortest path), a machine learning algorithm (splash belief propagation) and a logic puzzle (N queens).
\end{comment}

\subsection{Shortest Path}

The Single Source Shortest Path (SSSP) is a graph algorithm where we want to compute the
distance of all nodes to a single node. The full code is presented in Fig.~\ref{code:shortest_path_program}.
The graph is represented using the \texttt{edge(A, B, W)} predicate (line 1), where $A \rightarrow B$ is a directed edge
from node $A$ to $B$ with distance $W$. The second
predicate is \texttt{path(A, D, F)} (line 2), where $D$ is the distance of node $A$ to our initial node and $F$
indicates if this path has been processed.

Every node may have several \texttt{path} facts. From this set of facts, the node will select
the path with the least distance (with rule 1 and 2 in lines 11-13) and then propagate it using the third rule (lines 15-19).
When a path is used for propagation, we make sure that the path flag is \texttt{notused} (line 15)
because we do not want to propagate the same distance twice. When the rule is fired, the flag
then changes to \texttt{used} (line 19).

We start with the axiom $\texttt{path(startnode, 0, notused)}$ in line 9 (the distance to the starting node is 0).
This will kickstart the computation by propagating the initial distance to the neighbor nodes using
rule 3. We also use 2 coordination directives:

\begin{itemize}
   \item \texttt{priority @order asc}: paths are picked in ascending order (line 4).
   \item \texttt{set-priority(B, float(D + W))}: to change the node's priority upon the computation of
a new distance (line 18).
\end{itemize}

With these directives, we ensure that we pick the node with the smallest distance
first. If we are only using 1 thread, then our algorithm will behave like Dijkstra's shortest
path algorithm~\cite{Dijkstra}. When using more than 1 thread, each thread will pick the smallest
path from a subset of nodes. This is an example where using coordination directives will
improve the speed of execution since we avoid propagating some of the paths.

\begin{figure}[h!]
\small\begin{Verbatim}[numbers=left,commandchars=+\[\]]
type route edge(node, node, int).
type linear path(node, int, int).

+underline[priority @order asc].

const used = 1.
const notused = 0.

path(startnode, 0, notused).

path(A, B, used), path(A, B, notused) -o path(A, B, used).

path(A, B1, X), path(A, B2, Y), B1 <= B2 -o path(A, B1, X).

path(A, D, notused)
   -o {B, W | !edge(A, B, W) |
         path(B, D + W, notused),
         +underline[set-priority(B, float(D + W))]},
      path(A, D, used).
\end{Verbatim}
  \caption{Shortest Path Program.}
  \label{code:shortest_path_program}
\end{figure}
\normalsize

\subsection{Splash Belief Propagation}

Loopy Belief Propagation~\cite{Murphy99loopybelief} (LBP) is an approximate inference algorithm
used in graphical models with cycles. In its essence LBP is a sum-product message passing algorithm
where nodes exchange messages with their immediate neighbors and apply some computations to the messages
received.

LBP is an algorithm that maps very well to the \lang model. In its original form, we need to compute
the belief of all nodes for several iterations and also synchronize after each iteration.
However, it is still possible to apply
some optimizations in order to take advantage of the fact that LBP will converge even when using
an asynchronous approach. So, instead of computing the belief iteratively,
we first keep track of all messages sent/received (and overwrite them when we receive a new one)
and recompute the belief when we want, instead of synchronizing between nodes.
This way, we can prioritize the computation of beliefs when
a node's belief value changes significantly. When that happens, we set the priority of its
neighbors so that they can recompute their beliefs.

The asynchronous approach proves to be a nice improvement over the synchronous version. Still, it
is possible to do even better. Gonzalez et al~\cite{Gonzalez+al:aistats09paraml} developed an optimal
algorithm to compute this algorithm by first building a tree and then updating the beliefs of each node twice, first from the leaves to the root and then from the root to the leaves. The root of this tree
is the node with the highest priority (based on belief) while the other nodes in the tree
must have a non-zero priority. Note that the priorities are updated whenever a neighbor updates
their belief. These splash trees are built iteratively until we reach convergence.

The code in Fig.~\ref{code:sbp} presents the coordination component of the Belief Propagation problem.
Please note that we just appended the code in Fig.~\ref{code:sbp} to a working but unoptimized version
of the algorithm.
In the coordination code we have three sections:
\begin{description}
   \item[Tree building]: Each node has a \texttt{waiting} fact that is used to start the tree building process. When the highest priority is picked we create a token that will navigate through the tree. Note that in the rule located in lines 11-18 we check if the priority of the new node to add to the tree is positive and that both nodes are in the same processor. We want the tree to be kept in the same processor.
   \item[First phase]: In the third rule (lines 8-9), when we reach a certain number of nodes in the tree, we generate \texttt{first-phase} in order to update the beliefs of all nodes in tree starting from the leaves and ending at the root. As we update the nodes, we generate \texttt{update} to update the belief values (line 29).
   \item[Second phase]: In the second phase we go from the root to the leaves and update the beliefs a second time (line 39).
\end{description}

When we have several threads, every thread will generate their own trees by taking into account the highest priority node in their own queues. We do not use work stealing for this program because that would interfere with our coordination.

\begin{figure}[h!]
\small\begin{Verbatim}[numbers=left,commandchars=*\{\}]
// TREE BUILDING
// continue tree
waiting(A), token(A, All, Next) -o token(A, All, Next).
// start tree
waiting(A), *underline{@priority(A, A, P)}, P > 0.0
   -o token(A, [A], [A]).
// end tree building
token(A, All, Next), length(All) > maxnodes
   -o first-phase(A, All, reverse(All)).
// expand tree
token(A, All, [A | Next])
   -o [collect => L | Side | !edge(A, L, Side),
         0 = count(All, L),
         0 = count(Next, L),
         *underline{@priority(A, L, P)}, P > 0.0,
         *underline{@cpu-id(A, L, Id1)},
         *underline{@cpu-id(A, A, Id2)}, Id1 = Id2 |
         send-token(A, All, Next ++ L)].

send-token(A, All, [])
   -o first-phase(A, All, reverse(All)).
send-token(A, All, [B | Next])
   -o *underline{schedule-next(B)},
      token(B, All ++ [B], [B | Next]).

// FIRST PHASE
first-phase(A, [A], [A]) -o second-phase(A, [], A).
first-phase(A, [A, B | Next], [A])
   -o update(A), *underline{schedule-next(B)},
      second-phase(B, [B | Next], A).
first-phase(A, All, [A, B | Next])
   -o update(A), *underline{schedule-next(B)},
      first-phase(B, All, [B | Next]).

// SECOND PHASE
second-phase(A, [], _)
   -o *underline{set-priority(A, 0.0)}, waiting(A), update(A).
second-phase(A, [A], Back)
   -o update(A), waiting(Back),
      waiting(A), *underline{set-priority(A, 0.0)}.
second-phase(A, [A, B | Next], Back)
   -o update(A), waiting(Back), *underline{schedule-next(B)},
      second-phase(B, [B | Next], A).
\end{Verbatim}
  \caption{Coordination code for the Splash Belief Propagation. Note that a @ before a linear fact means it will be re-derived.}
  \label{code:sbp}
\end{figure}
\normalsize

%\begin{comment}
\subsection{N Queens}

The N queens~\cite{8queens} puzzle is the problem of placing N chess queens on an NxN chessboard so
that no pair of two queens attack each other. The specific challenge of finding all the distinct
solutions to this problem is a good benchmark in designing parallel algorithms.

We will not show the algorithm in this paper due to the lack of space, but we will explain in general
terms how we solve the problem in \lang. First, we consider each square of the chessboard as a node
that can communicate with the adjacent left, right and bottom squares, but not top square.
The states are represented as a list of integers, where each integer is the column number where
the queen was placed. For example $[2, 0]$ means that a queen was placed in $(0, 0)$ and $(1, 2)$.

An empty state is instantiated in the top-left node and is then propagated to all nodes in the same row.
Every node will then check if a queen can be placed on the square. If true, each node will send the new
state to the row below.
Recursively, when a node receives a new state, it will (1) send the state to the left
and to the right and (2) try to place the queen in its square. With this method,
all states will be computed since we have facts for each valid state
at that point. When a square cannot place a queen, that state is deleted.
When the program ends, the states will be placed in the bottom row.

We find our solution very elegant, since it can be easily executed in parallel and there are no
wasted computations, because any distinct placement will be computed only once.

\begin{comment}
\begin{figure}[h!]
\small\begin{verbatim}
type left(node, node).
type right(node, node).
type down(node, node).
type coord(node, int, int).
type linear propagate-left(node, list node, list int).
type linear propagate-right(node, list node, list int).
type linear receive-down(node, list node, list int).
type linear test-and-send-down(node, list node, list int).
type linear test-y(node, int, list int, list node, list int).
type linear test-diag-left(node, int, int, list int, list node, list int).
type linear test-diag-right(node, int, int, list int, list node, list int).
type linear send-down(node, list node, list int).
type linear final-state(node, list node, list int).

const size = 11.

receive-down(@0, [], []).

receive-down(A, Nodes, Coords)
   -o {R | !right(A, R), R <> A | propagate-right(R, Nodes, Coords)},
      {L | !left(A, L), L <> A | propagate-left(L, Nodes, Coords)},
      test-and-send-down(A, Nodes, Coords).

propagate-left(A, Nodes, Coords)
   -o {L | !left(A, L), L <> A | propagate-left(L, Nodes, Coords)},
      test-and-send-down(A, Nodes, Coords).

propagate-right(A, Nodes, Coords)
   -o {R | !right(A, R), R <> A | propagate-right(R, Nodes, Coords)},
      test-and-send-down(A, Nodes, Coords).

test-and-send-down(A, Nodes, Coords),
!coord(A, X, Y)
   -o test-y(A, Y, Coords, Nodes, Coords).

test-y(A, Y, [], Nodes, Coords), !coord(A, OX, OY) -o test-diag-left(A, OX - 1, OY - 1, Coords, Nodes, Coords).
test-y(A, Y, [X, Y1 | RestCoords], Nodes, Coords), Y = Y1 -o 1. // fail
test-y(A, Y, [X, Y1 | RestCoords], Nodes, Coords), Y <> Y1 -o test-y(A, Y, RestCoords, Nodes, Coords).

test-diag-left(A, X, Y, _, Nodes, Coords),
X < 0 || Y < 0,
!coord(A, OX, OY)
   -o test-diag-right(A, OX - 1, OY + 1, Coords, Nodes, Coords).

test-diag-left(A, X, Y, [X1, Y1 | RestCoords], Nodes, Coords),
X = X1, Y = Y1
   -o 1. // fail

test-diag-left(A, X, Y, [X1, Y1 | RestCoords], Nodes, Coords),
X <> X1 || Y <> Y1
   -o test-diag-left(A, X - 1, Y - 1, RestCoords, Nodes, Coords).

test-diag-right(A, X, Y, [], Nodes, Coords),
X < 0 || Y >= size,
!coord(A, OX, OY)
   -o send-down(A, [A | Nodes], [OX, OY | Coords]).

test-diag-right(A, X, Y, [X1, Y1 | RestCoords], Nodes, Coords),
X = X1, Y = Y1
   -o 1. // fail

test-diag-right(A, X, Y, [X1, Y1 | RestCoords], Nodes, Coords),
X <> X1 || Y <> Y1
   -o test-diag-right(A, X - 1, Y + 1, RestCoords, Nodes, Coords).

send-down(A, Nodes, Coords),
!down(A, A)
   -o final-state(A, Nodes, Coords).
   
send-down(A, Nodes, Coords),
!down(A, B),
A <> B
   -o receive-down(B, Nodes, Coords).
\end{verbatim}
  \caption{Visit program.}
  \label{code:visit}
\end{figure}
\normalsize
%\end{comment}
\end{comment}